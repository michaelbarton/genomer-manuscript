% Template for PLoS
% Version 1.0 January 2009
%
\documentclass[10pt]{article}

% cite package, to clean up citations in the main text. Do not remove.
\usepackage{cite}

\usepackage{color} 

% Use doublespacing - comment out for single spacing
%\usepackage{setspace} 
%\doublespacing

% Text layout
\topmargin 0.0cm
\oddsidemargin 0.5cm
\evensidemargin 0.5cm
\textwidth 16cm 
\textheight 21cm

% Bold the 'Figure #' in the caption and separate it with a period
% Captions will be left justified
\usepackage[labelfont=bf,labelsep=period,justification=raggedright]{caption}

% Use the PLoS provided bibtex style
\bibliographystyle{article}

% Remove brackets from numbering in List of References
\makeatletter
\renewcommand{\@biblabel}[1]{\quad#1.}
\makeatother

% Leave date blank
\date{}

\pagestyle{myheadings}
%% ** EDIT HERE **


%% ** EDIT HERE **
%% PLEASE INCLUDE ALL MACROS BELOW

%% END MACROS SECTION

\begin{document}

% Title must be 150 characters or less
\begin{flushleft}
{\Large
\textbf{}
}
% Insert Author names, affiliations and corresponding author email.
\\
Michael D. Barton$^{1\ast}$, 
Hazel A. Barton$^{1}$
\\
\bf{1} Biology Department, The University of Akron, Akron, OH, 44325-3908, USA
\\
$\ast$ E-mail: mail@michaelbarton.me.uk
\end{flushleft}

\section*{Abstract}

The increasing accessibility and reduced costs of sequencing has made genome
analysis accessible to more and more researchers. In contrast there remains a
steep learning curve in the subsequent computational steps to required to
process raw sequence reads into a database-deposited genome sequence. Here we
describe ``Genomer,'' a tool that simplifies the manual tasks of finishing and
uploading a genome sequence to a database. Genomer can format a genome scaffold
into the common files required to submit to GenBank. Genomer simplifies editing
a genome scaffold by allowing an human-readable YAML format file to be edited
instead of large sequence files. Genomer is written as a command line tool and
is an effort to make the manual, and increasingly common, process of genome
scaffolding more robust and reproducible. Extensive documentation and video
tutorials are available at http://next.gs.

\section*{Introduction}

The decreasing costs and increasing diversity of high-throughput sequencing
methods is making genome analysis a common method to tackle outstanding
questions in microbiology \cite{loman2012b}. Sequencing can produce
$10e^{6}$-$10e^{10}$ short nucleotide reads, which must be assembled into
larger contigs then scaffolded into larger megabase fragments or complete
chromosomes. These larger DNA sequences are then annotated with genomic
features such as protein-coding genes. Each of these steps can provide
imperfect results and remains the subject of active research
\cite{earl2011,quail2012,beckloff2012}, and manual curation by a researcher may
still improve upon these automated approaches.

Manually editing a genome sequence or annotation is a non-trivial problem
requiring significant effort on a researcher's part. The large size of FASTA
sequence files or general feature format (GFF) annotation files makes changing
these files in a text editor difficult. Tasks such as organising project files,
adding additional sequences to an assembly, updating incorrect annotations or
preparing files for database deposition all divert effort from the main goal of
analysing a genome. As a result genome projects have a steep learning curve for
researchers embarking on genome sequencing for the first time, while still
requiring effort for experienced researchers.

We anticipate these problems have resulted in individual groups independently
producing in-house programming scripts to automate repetitive tasks. An example
of such tasks include finding protein names that do not match a convention of
three lower case characters followed by one upper case character (abcD). This
is manually laborious but simple to automate computationally. Individual
research groups producing their own scripts for such tasks however leads to a
repetition of effort. In contrast open-source projects can lead to a pooling of
community efforts resulting in a higher standard of software and preventing
\cite{ince2012}.

During the process of sequencing several \emph{Pseudomonas fluorescens} strains
we developed a tool, called ``Genomer'', to ease and simplify many of the
common steps in a genome project. Several graphical user interface tools are
available to work with genome projects \cite{tanaka2006, wilkinson2002,
lopez2011, carver2012, gordon2003}. Genomer instead provides a simple, command
line interface for genome finishing allowing scriptable automation and
reproducible results. We have open-sourced this tool under and MIT license and
here we describe its application for simplifying steps in a microbial genome
project.

\section*{Implementation}

Genomer is written in the Ruby programming language \cite{ruby-lang,goto2010}
and tested against the 1.8.7 and 1.9.3 versions. Genomer is built upon the
Rubygems and Bundler Ruby package management libraries. Rubygems is a software
management system for Ruby which allows automatic downloading and installation
of third-party software libraries from the rubygems.org website. Bundler is
software for determining which specific versions of the available third
libraries should be used.

Genomer uses these two libraries to create a plugin system where third-party
software can be written and included. Plugin creators need only to prefix the
name of their plugin with `genomer-plugin-' and upload this to rubygems.org or
make the plugin available as a public git repository. A plugin is specified
within the project in a file named `Gemfile.' The `bundle update` command will
then automatically download and install the required plugins.

Genomer is implemented as a command line tool and tested on both Mac OS X and
Linux systems. The latest version of genomer can installed from the Rubygems
package management system with the command `gem install genomer`. Genomer is
then invoked on the command line with the command `genomer`. Genomer provides
extensive documentation via man pages and on the genomer website at
http://next.gs.

\section*{Results and Discussion}

Genomer was developed during our \emph{P. fluorescens} genome projects
\cite{barton2012} to automate and simplify the manual steps required when
finishing a microbial genome. Genomer includes functions to simplify the
following: updating and modifying a genome sequence, editing genome
annotations, and preparing the required files to submit a draft or complete
genome to a sequence database such as GenBank. Genomer is also designed to
allow researchers to add their own functionality to a genome project through
third-party plugins.

This software was written for the command-line to use in shells such as bash or
zsh. Command line tools are simple to automate using scripting, allowing
reproducible genome finishing and sharing genome projects between researchers.
For instance, during our microbial genome project we automated Genomer using
GNU Make build files, allowing the finishing steps to be repeated automatically
using the `make` command. Additionally we stored each version of the GNU Make
build file with git \cite{git-scm} to track changes to the project and to allow
errors to be reverted from earlier versions. Example build files from our
genome projects can be found on GitHub for a simple plasmid
\cite{plasmid-github} and more complex \cite{genome-github} sequence.

We divided this tool into modular parts each called a `plugin.' Genomer
provides no functionality itself, but rather serves to make the functionality
of plugins available for use at the command line. For example to validate the
correctness of genome annotations, the genomer-plugin-validate plugin should be
installed which then allows the required functionality to be through the
`genomer` command. Presently, we have developed and released three plugins:
viewing a genome sequence in common bioinformatics file formats, validating the
correctness of genome annotations for common errors, and provide summaries of a
genome sequence. Each of these plugins are available on GitHub and their use
described is in the documentation at http://next.gs.

The Genomer plugin system was built using the existing Ruby package management
system: RubyGems and Bundler. This allows plugins to be automatically installed
without requiring manual downloading and compilation by the user. We anticipate
this will eliminate a common problem where software may require technical
expertise to first compile and install before being used. A plugin system also
allows each plugin to be developed on different release schedules without
requiring Genomer itself to be updated, thereby providing a stable interface.
Each Genomer plugin can be locked to specific versions to prevent changes in
functionality between different plugin versions. This addresses situations
whereby backwards incompatible software changes break an existing genomics when
a recent version is available.

We have released Genomer as open source on GitHub \cite{genomer-github} and
have prepared documentation and video tutorials at http://next.gs. As we have
used this tool extensively to simplify our own genome projects we believe
genomer may also be useful to others in the field. A practice described as
``eating your own dog food'' \cite{harrison2006}. We will illustrate the
possible usefulness of genomer with two use cases: updating the annotations of
a genome scaffold during the finishing process, and preparing genome files for
submission to Genbank.

\subsection*{First Use Case: Continuous Genome Finishing}

Genome annotation is required to identify the features responsible for
phenotype. Online tools such as the Integrated Microbial Genomes system
\cite{markowitz2006} provide an online service whereby a nucleotide sequence
may be submitted and a subsequent file of predicted genomic features returned.
These annotations are required when submitting a genome sequence to a database.
Once annotated modifying the sequence becomes problematic as this requires
corresponding changes in all downstream annotated coordinates. This therefore
presents a problem whereby either no more updates can be made to the sequence
or the delta difference in annotation coordinates must be propagated to all
downstream locations. The former option is often choosen as being simpler and
therefore the genome sequence must be considered `frozen' once annotated and no
more changes can be made.

During our own \emph{P. fluorescens} genome sequencing projects, we aimed to
continuously improve the quality of the genome sequence by closing gaps using
PCR and Sanger sequencing while at the same time analysing the observed genomic
features. We therefore incorporated functionality into Genomer to allow
annotation coordinates to be automatically updated following such changes to
the sequence. Genomer allows this by organising a genome project as follows:

\begin{enumerate}

\item{Sequencing reads are assembled into contigs with a high degree of
confidence. This is done by using various assembly software and testing
different software parameters \cite{earl2011}. The generated contigs are then
compared between assembly replicates using nucmer from the Mummer software
package \cite{kurtz2004}. The FASTA file of contig sequences are then added to
the Genomer project.}

\item{The assembled contigs are then annotated and the GFF3 file of contig
annotations is added to the Genomer project.}

\item{A genome scaffold is then created using the scaffolder file format
\cite{barton2012}. This specifies the order and orientation of contigs in the
scaffold allowing any unresolved gap regions to also be specified. Paired-end
sequencing, comparison to reference genomes, or PCR amplification can be used
to estimate the positions of observed contigs in the true genome sequence.}
  
\item{The genomer-plugin-view plugin is then added to the Gemfile in the root
of the project directory. The command `bundle update`, run in the project root,
will then download and install this plugin making it available for use in the
Genomer project.}
  
\item{Different views can then be generated from the scaffold using the
`genomer view` command. For instance, a GFF3 file of the scaffolded annotations
may be generated. Should the originating contig annotations or the genome
scaffold be changed then rerunning the `genomer view` command will
correspondingly update the annotation coordinates for the scaffold.}

\end{enumerate}
  
This allows the scaffold file to be continuously improved even when already
annotated. For instance, additional PCR sequences can be used to close gaps and
improve the genome sequence, without requiring reannotation or manually
updating all annotation coordinates.

\subsection*{Second Use Case: Preparing files for submission to Genbank}

Generating a usable genome sequence requires sequencing, assembly of reads, and
annotation. A further step is also required: preparation of genomic files for
deposition in a database. An example is the Genbank database which requires the
genome annotations to be prepared in a custom table format. Additional files
are also required when the genome assembly is incomplete: a FASTA file
containing the individual contig sequences and `a golden path' (AGP)
\cite{agp-spec} file which describes the placement and direction of constituent
contigs in the draft genome sequence. The requirements to generate these
specific files can be time-consuming to complete and difficult to understand
for researchers new to field. We therefore Genomer during our own genome
projects to simplify and automate this process.

The genomer-view-plugin can be used to generate for submission to GenBank:
assembled FASTA, contig FASTA, annotation tables, and scaffold layout in AGP.
This includes FASTA sequences for both the scaffolded and individual contigs, a
GenBank table file of contig annotations, and an AGP file for the order of
contigs in the scaffold. Additional options may also be used to add a prefix to
gene annotation IDs, or begin locus ID numbering at the origin of the sequence,
thereby simplifying the process of creating the required files for submission.

Genomer does not produce the .asn which must be submitted to GenBank, instead
the generated files are used as the input for the tbl2asn or sequin tools. We
emphasise that the purpose of Genomer is not to reinvent the functionality
already provided by current NCBI resources, but instead to simplify the process
of producing the required inputs. Our example \emph{P. fluorescens} GNU Make
file [genome][] illustrates how Genomer may be combined with tbl2asn to produce
the required .asn file for submission.

\subsection*{Summary}

Genomer is an open-source tool to simplify and automate repetitive and
time-consuming tasks required when finishing a microbial genome project. This
software is available on GitHub \cite{genomer-github} with documentation and
video tutorials on the http://next.gs website. This tool has been useful in our
own genome projects and we believe it will also be useful to other researchers
especially smaller research groups entering the field for the first time.

% Do NOT remove this, even if you are not including acknowledgments
\section*{Acknowledgments}

This work was supported by the National Institute for Health: IDeA Network of
Biomedical Research Excellence (KY-INBRE) grant (NIH 2P20 RR016481-09) and the
NIH R15 AREA Program grant (R15GM079775).

%\section*{References}
% The bibtex filename
\bibliography{article}

\end{document}
