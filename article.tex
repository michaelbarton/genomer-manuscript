% Template for PLoS
% Version 1.0 January 2009
%
\documentclass[10pt]{article}

% cite package, to clean up citations in the main text. Do not remove.
\usepackage{cite}

\usepackage{color} 

% Use doublespacing - comment out for single spacing
%\usepackage{setspace} 
%\doublespacing

% Text layout
\topmargin 0.0cm
\oddsidemargin 0.5cm
\evensidemargin 0.5cm
\textwidth 16cm 
\textheight 21cm

% Bold the 'Figure #' in the caption and separate it with a period
% Captions will be left justified
\usepackage[labelfont=bf,labelsep=period,justification=raggedright]{caption}

% Use the PLoS provided bibtex style
\bibliographystyle{article}

% Remove brackets from numbering in List of References
\makeatletter
\renewcommand{\@biblabel}[1]{\quad#1.}
\makeatother

% Leave date blank
\date{}

\pagestyle{myheadings}
%% ** EDIT HERE **


%% ** EDIT HERE **
%% PLEASE INCLUDE ALL MACROS BELOW

%% END MACROS SECTION

\begin{document}

% Title must be 150 characters or less
\begin{flushleft}
{\Large
\textbf{Genomer - a Swiss army knife for genome scaffolding.}
}
% Insert Author names, affiliations and corresponding author email.
\\
Michael D. Barton$^{1\ast}$, 
Hazel A. Barton$^{1}$
\\
\bf{1} Biology Department, The University of Akron, Akron, OH, 44325-3908, USA
\\
$\ast$ E-mail: mail@michaelbarton.me.uk
\end{flushleft}

\section*{Abstract}

The increasing accessibility and reduced costs of sequencing has made genome
analysis accessible to more and more researchers. Yet there remains a steep
learning curve in the subsequent computational steps required to process raw
sequence reads into a database-deposited genome sequence. Here we describe
``Genomer,'' a tool to simplify the manual tasks of finishing and uploading a
genome sequence to a database. Genomer can format a genome scaffold into the
common files required for submission to GenBank. This software also simplifies
updating a genome scaffold by allowing a human-readable YAML format file to be
edited instead of large sequence files. Genomer is written as a command line
tool and is an effort to make the manual, and increasingly common, process of
genome scaffolding more robust and reproducible. Extensive documentation and
video tutorials are available at http://next.gs.

\section*{Introduction}

The decreasing costs and increasing diversity of high-throughput sequencing
methods is making genome analysis a common method to tackle unresolved
questions in microbiology \cite{loman2012b}. Sequencing can produce
$10e^{6}$-$10e^{10}$ short nucleotide reads, which must be assembled into
larger contigs which are then scaffolded into larger megabase fragments or
complete chromosomes. These larger DNA sequences are then annotated with
genomic features such as protein-coding genes. Each of these required steps can
however produce imperfect results and remains the subject of active research
\cite{earl2011,quail2012,beckloff2012}. Therefore manual curation of a genome
sequence and corresponding annotations may still improve upon the results of
automated methods.

Manually editing a genome sequence or annotation is a non-trivial problem
requiring effort on a researcher's part. The large size of FASTA sequence files
or general feature format (GFF) annotation files makes changing these in a text
editor difficult. Tasks such as organising project files, adding additional
sequences to an assembly, updating incorrect annotations or preparing files for
database deposition all divert effort from the main goal of analysing a genome.
As a result there is a steep learning curve for researchers embarking on genome
sequencing for the first time, while still requiring effort from experienced
researchers.

We anticipate these problems have resulted in individual groups independently
producing in-house programming scripts to automate repetitive tasks. An example
of such tasks include finding protein names that do not match a convention of
three lower case characters followed by one upper case character (abcD). Such
tasks are manually laborious but simple to automate computationally. Individual
research groups producing their own scripts for such tasks however leads to a
repetition of effort. In contrast open-source projects can lead to a pooling of
community effort resulting in a higher standard of software and preventing
reinvention of the same code \cite{ince2012}.

During the process of sequencing several \emph{Pseudomonas fluorescens} strains
we developed a tool, called ``Genomer,'' to ease and simplify many of the
common steps in our genome project. While several graphical user interface
tools are available to work with genome projects, including G-InforBio,
Genquire, Pile-LineGUI, Artemis and Consed \cite{tanaka2006, wilkinson2002,
lopez2011, carver2012, gordon1998}. Genomer provides a simple command line
interface which allows scriptable automation and reproducible results. We made
this software open-source under the MIT license and here we describe its
application.

\section*{Implementation}

Genomer is written in the Ruby programming language (1.8.7 and 1.9.3)
\cite{ruby-lang,goto2010} and built upon the Rubygems and Bundler Ruby package
management libraries. Rubygems is a software management system for Ruby which
allows automatic downloading and installation of third-party software libraries
from the rubygems.org website. Bundler is software for determining which
specific versions of the available third libraries should be used.

Genomer uses these two libraries to create a plugin system where third-party
software can be written and included. Plugin creators need only to prefix the
name of their plugin with \verb+genomer-plugin-+ and upload this to
rubygems.org or make the plugin available as a public git repository. A plugin
is specified within a Genomer project in a file named `Gemfile.' The
\verb+bundle update+ command will then automatically download and install the
required plugins plus any gem dependencies these gems rely on.

Genomer is implemented as a command line tool and tested on both Mac OS X and
Linux systems. The latest version of Genomer can installed from the Rubygems
package management system with the command \verb+gem install genomer+. Genomer
is based on our previously described ``Scaffolder'' software \cite{barton2012}
and is invoked on the command line with the command \verb+genomer+. Genomer
provides extensive documentation via UNIX manual pages and on the Genomer
website at http://next.gs.

\section*{Results and Discussion}

Genomer was developed during our \emph{P. fluorescens} genome projects to
automate and simplify the manual steps required when finishing a microbial
genome. Genomer provides the following functionality to facilitate this:

\begin{description} 

  \item[Simple editing of a draft genome sequence.]{Genomer is built on the
  existing Scaffolder \cite{barton2012} file format for assembling draft genome
  sequences. This is format requires only the order and IDs of each contig be
  specified. This thereby simplifies the process of re-organising and trimming
  contigs in draft genome without having to edit large nucleotide sequences
  manually.}

  \item[Mapping of annotations onto the assembled sequence.]{Genomer maps the
  coordinates of contig annotations to their respective positions in the draft
  genome sequence. This allows the scaffold file to be continuously improved
  and updated even after the contigs have been annotated. For instance,
  additional PCR sequences can be used to close gaps in the assembly and the
  location of all annotations be correspondingly updated.}

  \item[Generation of files for submission to GenBank.]{Submitting a genome
  sequence to GenBank database requires generating specific files. These may
  include a FASTA file of the draft sequence, a table of annotations, a FASTA
  file containing the individual contig sequences, and `a golden path' (AGP)
  \cite{agp-spec} file describing the placement of contigs. Genomer automates
  the generation of all these files from the scaffold file and corresponding
  GFF3 file and removes the effort to produce these manually.}

  \item[A stable interface and streamlined install process]{Genomer provides a
  plugin system built using the existing Ruby package management system:
  RubyGems and Bundler. This allows Genomer and its plugins to be automatically
  installed without requiring manual downloading and compilation by the user.
  This eliminates a common problem in bioinformatics where software may require
  technical expertise to first compile and install before being used. Install
  plugins may also be locked to specific versions to prevent backwards
  incompatible software changes breaking an existing workflow.}

  \item[Integration with the command line.]{Genomer is built as a command line
  tool around plain text files. This allows for integration with common Linux
  tools tools such as GNU Make or git thereby allowing reproducible scripting
  of the genome finishing process.}

\end{description}

Genomer is written as a command-line tool for use in shells such as bash or
zsh. The advantage of command-line tools over GUI-based tools, is that the former
are simpler to automate and allow the sharing of steps with other researchers
working on related projects. For instance, during our microbial genome project
we automated Genomer using GNU Make build files, allowing the finishing steps
to be repeated automatically using the \verb+make+ command. Genomer may
therefore be of particular interest to bioinformaticians coordinating in groups
on the same genome project and those who prefer automated approaches and
scripting on the command line.

As Genomer uses plain text files to manage the draft genome sequence it is easy
to store versions of the project using a revision control system such as git
\cite{git-scm}. This allows tracking changes to the project and reverting
errors back to earlier versions. Example versioned build files from our genome
projects can be found on GitHub for a simple plasmid \cite{plasmid-github} and
a more complex genome \cite{genome-github} sequence.

We have released Genomer as open source software on GitHub
\cite{genomer-github} and have prepared documentation and video tutorials at
http://next.gs. We have used this tool extensively to simplify our own genome
projects and believe Genomer may also be useful to others in the field. We will
illustrate the possible application of Genomer with the following example use
case.

\subsection*{Genomer use case}

\begin{enumerate}

  \item{A Genomer project is organised around a set of already assembled and
  annotated contigs. In our microbial genome projects we assemble our reads
  into contigs using the A5 pipeline \cite{tritt2012} and then subsequently
  annotate them using the Integrated Microbial Genomes resource
  \cite{markowitz2006}. We then use these contigs as a starting point to build
  a draft genome sequence in Genomer.}

  \item{A genome scaffold is written in the scaffolder file format
  \cite{barton2012}. This specifies the order and orientation of contigs and
  unresolved gap regions in the scaffold. Paired-read sequencing or comparison
  to reference genomes can be used as a source of contig orientation and order.
  As an example, in our own genome projects we determine contig order by
  aligning contigs to a reference genome using nucmer \cite{kurtz2004}. A
  detailed overview of the scaffolder file format can be found on the Genomer
  website.}

  \item{The \verb+genomer view+ command is used to generate files of the
  assembled scaffold, such as the FASTA file of the assembly or a GFF3 file of
  annotations locations on the assembled genome sequence. These files can then
  be used for subsequent analyses of the sequence.}

  \item{The genome scaffold can be continuously updated through closing any
  remaining gaps. During our microbe sequencing projects we closed gaps through
  a combination of Sanger PCR and \emph{in silico} analysis. Genomer allows
  changes in the scaffold file to automatically propagate to the required
  changes in the generated files when the \verb+genomer view+ command is
  rerun.}

  \item{The required files for submission to GenBank may be generated when
  researchers are satisfied with the status of their draft. These files include
  the assembled FASTA file, contig FASTA file, annotation table file, and AGP
  scaffold file. Additional options can be used to add the required prefix to
  gene annotation IDs and begin locus ID numbering at the origin of the
  sequence. Genomer does not produce the .asn file which is submitted to
  GenBank, but instead the files that may be used as the input to tbl2asn or
  sequin provided by GenBank.}

\end{enumerate}

\subsection*{Limitations}

Genomer was written to satisfy our needs for building and finishing megabase
sized draft genomes where assembly of the draft sequence takes place in memory.
This makes Genomer well-suited for microbial genome projects where
approximately 99\% of microbial genomes sequenced to date are less than 10MBp.
The Assembly of very large sequences may however limit performance when the
size of the sequence exceeds available. We however anticipate that genome sizes
of even 3GBp will fit in the memory of a modern laptop or desktop computer.

One features of Genomer is the simplification of generating the annotation
table file required for submission to GenBank. Genomer requires only a GFF3
file containing gene-type entries be required. Genomer is able to generate the
additional encoded protein features from these entries. One caveat to this
however is that the assumption there is no intron/exon structure and therefore
users wishing to submit annotations with alternative splicing cannot take
advantage of this Genomer feature.

\subsection*{Summary}

Genomer is an open-source tool to simplify and automate repetitive and
time-consuming tasks required when finishing a microbial genome project. This
software is available on GitHub \cite{genomer-github} with documentation and
video tutorials on the http://next.gs website. This tool has been useful in our
own genome projects and we believe it will also be useful to other researchers
especially smaller research groups entering the field for the first time.

% Do NOT remove this, even if you are not including acknowledgments
\section*{Acknowledgements}

This work was supported by the National Institute for Health: IDeA Network of
Biomedical Research Excellence (KY-INBRE) grant (NIH 2P20 RR016481-09) and the
NIH R15 AREA Program grant (R15GM079775).

%\section*{References}
% The bibtex filename
\bibliography{article}

\end{document}
